\documentclass[11pt,twocolumn]{article} % Changed to article class for paper-like format

\usepackage[utf8]{inputenc}
\usepackage{graphicx}
\usepackage{amsmath}
\usepackage{amssymb}
\usepackage{hyperref}
\usepackage{geometry}
\geometry{a4paper, margin=1in}
\usepackage{caption}
\usepackage{subcaption}
\usepackage{datetime}
\usepackage{listings}
\usepackage{float} % For better image placement

\newdateformat{monthyear}{\monthname[\THEMONTH] \THEYEAR}

% Helper macros for derivatives
\newcommand{\ddt}[1]{\frac{d}{dt}#1}
\newcommand{\dddt}[1]{\frac{d^2}{dt^2}#1}

\begin{document}

\begin{titlepage}
    \centering
    \includegraphics[width=0.15\textwidth]{iitm.png}\par\vspace{1cm}
    
    {\scshape\LARGE Indian Institute of Technology Madras \par}
    \vspace{1.5cm}
    
    {\LARGE \bfseries Control of Ball Balancing Robot \par}
    \vspace{1cm}
    
    {\Large\bfseries Course: ME4010 \par}
    \vspace{0.5cm}
    
    {\Large Professor: Dr. Manish Anand \par}
    \vspace{2cm}
    
    {\large \textbf{Authors:} \par}
    \vspace{0.5cm}
    {\large 
    T. Abhinand (ME23B208)\\
    }
    
    \vfill
    {\large \monthyear\today\par}
\end{titlepage}

%\maketitle % Removed as it's often not used with a custom titlepage in article class
%\tableofcontents % Removed as per user's request for simpler format. If needed, can be re-added as \section*{Contents}
\newpage

\section{Introduction}
This report details the control strategies applied to a ball balancing robot, focusing on the derivation of its equations of motion and the implementation of LQR and Root Locus control techniques. The objective is to maintain the stability of a ball on a moving platform.

\section{System Modeling}

\subsection{System Description}
The system consists of a ball balancing robot modeled with three degrees of freedom. The mechanical structure is defined by the interaction between the wheel, the robot body, and a balancing mass.

\begin{figure}[h!]
    \centering
    \includegraphics[width=\columnwidth]{system.png} % Adjusted for two-column layout
    \caption{Diagram of the Ball Balancing Robot System}
    \label{fig:system_diagram}
\end{figure}

The system parameters and generalized coordinates are defined as follows:

\begin{table}[h!]
\centering
\begin{tabular}{|c|l|}
\hline
\textbf{Symbol} & \textbf{Description} \\ \hline
\(x_w(t)\) & Linear displacement of the wheel/cart \\ \hline
\(\theta(t)\) & Tilt angle of the robot body \\ \hline
\(x(t)\) & Position of the balancing mass relative to the body \\ \hline
\(M_w, I_w\) & Mass and Inertia of the wheel \\ \hline
\(M_r, I_r\) & Mass and Inertia of the robot body \\ \hline
\(m_b, I_b\) & Mass and Inertia of the balancing mass \\ \hline
\(R\) & Radius of the wheel \\ \hline
\(r\) & Radius of the ball \\ \hline
\(d\) & Distance to the robot body center of mass \\ \hline
\(h\) & Height parameter for the balancing mass \\ \hline
\(g\) & Gravitational acceleration \\ \hline
\end{tabular}
\caption{System Parameters}
\label{tab:params}
\end{table}

\subsection{Equations of Motion Derivation}
The equations of motion are derived using the Euler-Lagrange method. The Lagrangian \(\mathcal{L}\) is defined as the difference between the total kinetic energy (\(T\)) and the total potential energy (\(U\)) of the system.

\subsubsection{Kinetic Energy (\(T\))}
The total kinetic energy is the sum of the kinetic energies of the wheel (\(T_w\)), the robot body (\(T_r\)), and the balancing mass (\(T_b\)).

\begin{equation}
    T_w = \frac{1}{2} M_w \dot{x}_w^2 + \frac{1}{2} I_w \left(\frac{\dot{x}_w}{R}\right)^2
\end{equation}

\begin{equation}
    T_r = \frac{1}{2} M_r \dot{x}_w^2 + \frac{1}{2} I_r \dot{\theta}^2
\end{equation}

The kinetic energy of the balancing mass \(T_b\) includes both translational and rotational components, accounting for the relative motion \(x(t)\) and the body angle \(\theta(t)\):
\begin{equation}
    T_b = \frac{1}{2} m_b (v_{b,x}^2 + v_{b,y}^2) + \frac{1}{2} I_b \left(\frac{\dot{x}}{r}\right)^2
\end{equation}

\subsubsection{Position and Velocity Analysis}

The position of the balancing mass in the inertial frame is derived by first expressing its position in the body-fixed frame, then applying a rotation and translation.

\paragraph{Body-Fixed Coordinates}
In the body-fixed frame, the position of the balancing mass relative to the body center is:
\begin{equation}
    \begin{bmatrix} x_{rel} \\ y_{rel} \end{bmatrix} = \begin{bmatrix} x \\ h + r \end{bmatrix}
\end{equation}

\paragraph{Rotation Matrix}
The rotation from the body-fixed frame to the inertial frame is represented by the rotation matrix:
\begin{equation}
    \mathbf{R}(\theta) = \begin{bmatrix} \cos(\theta) & \sin(\theta) \\ -\sin(\theta) & \cos(\theta) \end{bmatrix}
\end{equation}

where \(\theta\) is the tilt angle of the robot body.

\paragraph{Inertial Position Derivation}
The position of the balancing mass in the inertial frame is obtained by rotating the body-fixed coordinates and adding the wheel position offset:

Performing the matrix multiplication:
\begin{equation}
    \begin{bmatrix} x_b \\ y_b \end{bmatrix} = \begin{bmatrix} \cos(\theta) & \sin(\theta) \\ -\sin(\theta) & \cos(\theta) \end{bmatrix} \begin{bmatrix} x \\ h + r \end{bmatrix} + \begin{bmatrix} x_w \\ 0 \end{bmatrix}
\end{equation}

Expanding the matrix product:
\begin{align}
    x_b &= \cos(\theta) \cdot x + \sin(\theta) \cdot (h + r) + x_w \\
    y_b &= -\sin(\theta) \cdot x + \cos(\theta) \cdot (h + r)
\end{align}

Rearranging:
\begin{align}
    x_b &= x\cos(\theta) + (h + r)\sin(\theta) + x_w \label{eq:x_b} \\
    y_b &= (h + r)\cos(\theta) - x\sin(\theta) \label{eq:y_b}
\end{align}

\subsubsection{Potential Energy (\(U\))}
The total potential energy is the sum of the potential energies of the components:
\begin{equation}
    U = U_b + U_r + U_w
\end{equation}
where:
\begin{align}
    U_b &= m_b g ((h + r)\cos(\theta) - x\sin(\theta)) + R + \text{const} \\
    U_r &= M_r g d \cos(\theta) + R \\
    U_w &= R \quad (\text{constant})
\end{align}

\subsubsection{Lagrangian}
The Lagrangian of the system is given by:
\begin{equation}
    \mathcal{L} = T - U
\end{equation}

\subsubsection{Equations of Motion}
Applying the Euler-Lagrange equation:
\begin{equation}
    \frac{d}{dt}\left(\frac{\partial \mathcal{L}}{\partial \dot{q}_i}\right) - \frac{\partial \mathcal{L}}{\partial q_i} = 0
\end{equation}
for the generalized coordinates \(q = [x_w, \theta, x]\), we obtain the three equations of motion:
% Placeholder for equations from equations_of_motion.ipynb
% These are the equations that would be extracted directly from the notebook's LaTeX output.
\begin{itemize}
    \item \textbf{Equation for \(x_w\):} (LaTeX output for EOM for x_w here)
    \item \textbf{Equation for \(\theta\):} (LaTeX output for EOM for theta here)
    \item \textbf{Equation for \(x\):} (LaTeX output for EOM for x here)
\end{itemize}


\subsection{State-Space Formulation}

\subsubsection{Mass Matrix}
The system dynamics can be expressed in the form:
\begin{equation}
    \mathbf{M}(\mathbf{q}) \ddot{\mathbf{q}} + \mathbf{K}(\mathbf{q}, \dot{\mathbf{q}}) = 0
\end{equation}

The mass/inertia matrix \(\mathbf{M}_1\) is:
\begin{equation}
    \mathbf{M}_1 = \begin{bmatrix}
        \frac{I_b}{r^2} + m_b & m_b(h + r) \\
        m_b(h + r) & I_r + m_b(h + r)^2
    \end{bmatrix}
\end{equation}

\subsubsection{Stiffness Matrix}
The stiffness/restoring force matrix \(\mathbf{M}_2\) captures the gravitational restoring terms:
\begin{equation}
    \mathbf{M}_2 = \begin{bmatrix}
        0 & g m_b \\
        g m_b & g(M_r d + m_b(h + r))
    \end{bmatrix}
\end{equation}

\subsubsection{Input Coupling Matrix}
The torque/input coupling matrix \(\mathbf{M}_3\) represents the effect of the motor torque on the generalized coordinates:
\begin{equation}
    \mathbf{M}_3 = \begin{bmatrix}
        -m_b(h + r) \\
        -I_r - m_b(h + r)^2
    \end{bmatrix}
\end{equation}

\subsubsection{Linearized State-Space Matrices}
The system can be linearized around the equilibrium point \(\theta = 0\) to obtain the linear state-space representation. Define the state vector as:
\begin{equation}
    \mathbf{x} = \begin{bmatrix} x \\ \theta \\ \dot{x} \\ \dot{\theta} \end{bmatrix}
\end{equation}

The linearized state-space dynamics are:
\begin{equation}
    \dot{\mathbf{x}} = \mathbf{A} \mathbf{x} + \mathbf{B} u
\end{equation}

where the state transition matrix \(\mathbf{A}\) is:
\begin{equation}
    \mathbf{A} = \begin{bmatrix}
        0 & 1 & 0 & 0 \\
        - \frac{g m_b}{I_b / r^{2} + m_b} & 0 & 0 & 0 \\
        0 & 0 & 0 & 1 \\
        \frac{g m_b (I_b / r^{2} + m_b) - g (M_r d + m_b (h + r)) (I_b / r^{2} + m_b)}{(I_b / r^{2} + m_b) (I_r + m_b (h + r)^{2}) - m_b^{2} (h + r)^{2}} & 0 & 0 & 0
    \end{bmatrix}
\end{equation}
And the input matrix \(\mathbf{B}\) is:
\begin{equation}
    \mathbf{B} = \begin{bmatrix} 0 \\ \frac{- m_b (h + r)}{I_b / r^{2} + m_b} \\ 0 \\ \frac{- (I_b / r^{2} + m_b) (I_r + m_b (h + r)^{2}) + m_b^{2} (h + r)^{2}}{(I_b / r^{2} + m_b) (I_r + m_b (h + r)^{2}) - m_b^{2} (h + r)^{2}} \end{bmatrix}
\end{equation}

The output equation is:
\begin{equation}
    y = \mathbf{C} \mathbf{x} + \mathbf{D} u
\end{equation}
where \(\mathbf{C} = \begin{bmatrix} 0 & 0 & 1 & 0 \end{bmatrix}\) and \(\mathbf{D} = 0\).

\section{Control and State Estimation}

\subsection*{Numerical Parameters} % Changed from \section* to \subsection*
\begin{table}[h!]
\centering
\begin{tabular}{|c|c|l|}
\hline
\textbf{Symbol} & \textbf{Value} & \textbf{Units} \\ \hline
$\mathbf{M_w}$ & 4.3 & Mass of wheel ($\mathbf{kg}$) \\ \hline
$\mathbf{M_r}$ & 10.12 & Mass of robot body ($\mathbf{kg}$) \\ \hline
$\mathbf{m_b}$ & 0.00271 & Mass of the ball ($\mathbf{kg}$) \\ \hline
$\mathbf{I_w}$ & 0.2725 & Inertia of one wheel ($\mathbf{kg \cdot m^2}$) \\ \hline
$\mathbf{I_r}$ & 0.4747 & Inertia of robot body ($\mathbf{kg \cdot m^2}$) \\ \hline
$\mathbf{I_b}$ & $1.740 \times 10^{-6}$ & Inertia of the ball ($\mathbf{kg \cdot m^2}$) \\ \hline
$\mathbf{R}$ & 0.356 & Radius of the wheel ($\mathbf{m}$) \\ \hline
$\mathbf{r}$ & 0.04006 & Radius of the ball ($\mathbf{m}$) \\ \hline
$\mathbf{d}$ & 0.18865 & Distance to robot body CoM ($\mathbf{m}$) \\ \hline
$\mathbf{h}$ & 0.31665 & Robot height ($\mathbf{m}$) \\ \hline
$\mathbf{g}$ & 9.81 & Gravitational acceleration ($\mathbf{m/s^2}$) \\ \hline
\end{tabular}
\caption{ Numerical Parameters for the Ball Balancing Robot System}
\label{tab:final_params}
</table
\end{table}

\section{Numerical State-Space Model}

Using the parameters in Table~\ref{tab:final_params}, the linearized state-space model
\(\dot{\mathbf{x}} = \mathbf{A}\mathbf{x} + \mathbf{B}u,\ y = \mathbf{C}\mathbf{x} + \mathbf{D}u\)
is evaluated numerically as:

\begin{equation}
    \mathbf{A} =
    \begin{bmatrix}
        0 & 1 & 0 & 0 \\
        0 & 0 & -2.4497 & 0 \\
        0 & 0 & 0 & 1 \\
        0 & 0 & 58.1065 & 0
    \end{bmatrix}
\end{equation}

\begin{equation}
    \mathbf{B} =
    \begin{bmatrix}
        0 \\ 0.0076 \\ 0 \\ -0.0041
    \end{bmatrix}
\end{equation}

\begin{equation}
    \mathbf{C} = \begin{bmatrix} 0 & 0 & 1 & 0 \end{bmatrix},
    \qquad
    \mathbf{D} = \begin{bmatrix} 0 \end{bmatrix}
\end{equation}

\section{LQR Controller Design}

\subsection{Optimal Control Problem}
The Linear Quadratic Regulator (LQR) is designed to stabilize the unstable ball balancing robot while minimizing a cost function:
\begin{equation}
    J = \int_0^{\infty} \left( \mathbf{x}^T \mathbf{Q} \mathbf{x} + \mathbf{u}^T \mathbf{R} \mathbf{u} \right) dt
\end{equation}

\subsection{Bryson's Rule for Weight Selection}
\begin{equation}
    \mathbf{Q} = \text{diag}\left[\frac{1}{x_{\max}^2}, \frac{1}{\dot{x}_{\max}^2}, \frac{1}{\theta_{\max}^2}, \frac{1}{\dot{\theta}_{\max}^2}\right],
    \qquad
    \mathbf{R} = \frac{1}{u_{\max}^2}
\end{equation}

\begin{align*}
    x_{\max} &= d/8 = 0.0236 \text{ m} \\
    \dot{x}_{\max} &= 0.5 \text{ m/s} \\
    \theta_{\max} &= 15^\circ = 0.2618 \text{ rad} \\
    \dot{\theta}_{\max} &= 2 \text{ rad/s} \\
    u_{\max} &= 5 \text{ (actuator units)}
\end{align*}

\subsection{Optimal Feedback Control Law}
\begin{equation}
    \mathbf{A}^T \mathbf{S} + \mathbf{S} \mathbf{A} - \mathbf{S} \mathbf{B} \mathbf{R}^{-1} \mathbf{B}^T \mathbf{S} + \mathbf{Q} = 0
\end{equation}
\begin{equation}
    \mathbf{K} = \mathbf{R}^{-1} \mathbf{B}^T \mathbf{S}, \qquad u = -\mathbf{K}\mathbf{x}
\end{equation}

\subsection{Closed-Loop System}
\begin{equation}
    \dot{\mathbf{x}} = (\mathbf{A} - \mathbf{B}\mathbf{K}) \mathbf{x} = \mathbf{A}_{cl} \mathbf{x}
\end{equation}

\begin{figure}[h!]
    \centering
    \includegraphics[width=\columnwidth]{controller_block.png} % Adjusted for two-column layout
    \caption{LQR Controller Block Diagram}
    \label{fig:controller}
\end{figure}

\section{State Estimation}

\subsection{Observability Analysis}
\begin{equation}
    \mathbf{O} = \begin{bmatrix} \mathbf{C} \\ \mathbf{C}\mathbf{A} \\ \mathbf{C}\mathbf{A}^2 \\ \mathbf{C}\mathbf{A}^3 \end{bmatrix}
\end{equation}

\section{Full-Order Kalman Observer}

\subsection{Observer Dynamics}
\begin{equation}
    \dot{\hat{\mathbf{x}}} = \mathbf{A}\hat{\mathbf{x}} + \mathbf{B}u + \mathbf{L}(y - \mathbf{C}\hat{\mathbf{x}})
\end{equation}

\subsection{Kalman Filter Gain Design}
\begin{equation}
    \mathbf{Q}_{obs} = 0.1 \cdot \text{diag}[10^7, 100, 10^6, 10], \qquad
    \mathbf{R}_{obs} = 0.00001
\end{equation}
\begin{equation}
    \mathbf{L} = \mathbf{P}\mathbf{C}^T \mathbf{R}_{obs}^{-1}
\end{equation}

\subsection{Estimation Error Dynamics}
\begin{equation}
    \dot{\mathbf{e}} = (\mathbf{A} - \mathbf{L}\mathbf{C})\mathbf{e}
\end{equation}

\begin{figure}[h!]
    \centering
    \includegraphics[width=\columnwidth]{full_observer_block.png} % Adjusted for two-column layout
    \caption{Full-Order Kalman Observer}
    \label{fig:full_observer}
\end{figure}

\begin{figure}[h!]
    \centering
    \includegraphics[width=\columnwidth]{full_obs_normal.png}
    \caption{Full-order observer response: Ideal case}
    \label{fig:full_obs_normal}
\end{figure}
\begin{figure}[h!]
    \centering
    \includegraphics[width=\columnwidth]{full_obs_noise.png}
    \caption{Full-order observer response: With measurement noise}
    \label{fig:full_obs_noise}
\end{figure}
\begin{figure}[h!]
    \centering
    \includegraphics[width=\columnwidth]{full_obs_noise_filtered.png}
    \caption{Full-order observer response: Noise filtered}
    \label{fig:full_obs_noise_filtered}
\end{figure}

\section{Minimum-Order (Reduced-Order) Observer}

\subsection{Motivation}
Since \(\theta\) is directly measured, a minimum-order observer estimates only unmeasured states \(\dot{x}, x, \dot{\theta}\).

\subsection{State Partitioning}
\begin{equation}
    \mathbf{x}_a = \begin{bmatrix} \theta \end{bmatrix},
    \qquad
    \mathbf{x}_b = \begin{bmatrix} x \\ \dot{x} \\ \dot{\theta} \end{bmatrix}
\end{equation}

\begin{equation}
    \begin{bmatrix} \dot{\mathbf{x}}_a \\ \dot{\mathbf{x}}_b \end{bmatrix} =
    \begin{bmatrix} \mathbf{A}_{aa} & \mathbf{A}_{ab} \\ \mathbf{A}_{ba} & \mathbf{A}_{bb} \end{bmatrix}
    \begin{bmatrix} \mathbf{x}_a \\ \mathbf{x}_b \end{bmatrix} +
    \begin{bmatrix} \mathbf{B}_a \\ \mathbf{B}_b \end{bmatrix} u
\end{equation}

\subsection{Reduced-Order Observer}
\begin{equation}
    \dot{\mathbf{z}} = \mathbf{A}_{bb} \mathbf{z} + \mathbf{A}_{ba} \mathbf{x}_a + \mathbf{B}_b u +
    \mathbf{K}_{min}(\mathbf{x}_a - \mathbf{A}_{aa} \mathbf{x}_a - \mathbf{A}_{ab} \mathbf{z} - \mathbf{B}_a u)
\end{equation}
\begin{equation}
    \hat{\mathbf{x}}_b = \mathbf{z} + \mathbf{K}_{min} \mathbf{x}_a
\end{equation}

\subsection{Noise Covariances}
\begin{equation}
    \mathbf{Q}_{min} = 10000 \cdot \text{diag}[1, 100, 1], \qquad
    \mathbf{R}_{min} = 0.005
\end{equation}

\begin{figure}[h!]
    \centering
    \includegraphics[width=\columnwidth]{reduced_observer_block.png} % Adjusted for two-column layout
    \caption{Minimum-Order Observer}
    \label{fig:reduced_observer}
</figure>

\begin{figure}[h!]
    \centering
    \includegraphics[width=\columnwidth]{min_obs_normal.png}
    \caption{Minimum-order observer response: Ideal case}
    \label{fig:min_obs_normal}
\end{figure}
\begin{figure}[h!]
    \centering
    \includegraphics[width=\columnwidth]{min_obs_noise.png}
    \caption{Minimum-order observer response: With measurement noise}
    \label{fig:min_obs_noise}
</figure>
\begin{figure}[h!]
    \centering
    \includegraphics[width=\columnwidth]{min_obs_noise_filtered.png}
    \caption{Minimum-order observer response: Noise filtered}
    \label{fig:min_obs_noise_filtered}
</figure>

\section{Minimum-Order Observer with Noise Filtering}

\subsection{Motivation for Filtering}
Measurement noise on \(\theta\) can degrade observer performance. A low-pass filter is applied to the measured angle before feeding it to the minimum-order observer.

\subsection{First-Order Low-Pass Filter (14 Hz)}
A first-order low-pass filter is implemented with cutoff frequency \(f_c = 14\) Hz:
\begin{equation}
    \tau = \frac{1}{2\pi f_c} = \frac{1}{2\pi \cdot 14} \approx 0.01137\ \text{s}
\end{equation}

The filter transfer function is:
\begin{equation}
    H(s) = \frac{1}{\tau s + 1} = \frac{1}{0.01137 s + 1}
\end{equation}

The filtered measurement \(y_{filt}\) is obtained from:
\begin{equation}
    \dot{y}_{filt} = -\frac{1}{\tau}(y_{filt} - y_{meas})
\end{equation}
or in discrete form:
\begin{equation}
    y_{filt}(k+1) = y_{filt}(k) + \frac{\Delta t}{\tau}\bigl(y_{meas}(k) - y_{filt}(k)\bigr)
\end{equation}

\subsection{Filter Characteristics}
The cutoff frequency of 14 Hz is chosen such that:
\begin{itemize}
    \item \textbf{High-frequency noise attenuation}: Measurement noise above 14 Hz is significantly attenuated
    \item \textbf{System dynamics preservation}: The robot's physical angle dynamics (3–6 Hz range) are fully preserved
    \item \textbf{Minimal phase lag}: At system frequencies, the filter introduces only a few milliseconds of phase lag
\end{itemize}

\subsection{Butterworth Filter Alternative (14 Hz)}
For a second-order Butterworth filter:
\begin{equation}
    H(s) = \frac{\omega_n^2}{s^2 + \sqrt{2}\,\omega_n s + \omega_n^2}
\end{equation}
with
\begin{equation}
    \omega_n = 2\pi f_c = 2\pi \cdot 14 \approx 87.96\ \text{rad/s}
\end{equation}

The discrete-time realization uses the bilinear transformation with filter coefficients:
\begin{equation}
    y_{filt}(k) = -a_1 y_{filt}(k-1) - a_2 y_{filt}(k-2) + b_0 y_{meas}(k) + b_1 y_{meas}(k-1) + b_2 y_{meas}(k-2)
\end{equation}

\subsection{Integrated Observation Architecture}
\begin{equation}
    \dot{\mathbf{z}} = \mathbf{A}_{bb} \mathbf{z} + \mathbf{A}_{ba} y_{filt} + \mathbf{B}_b u +
    \mathbf{K}_{min}(y_{filt} - \mathbf{A}_{aa} y_{filt} - \mathbf{A}_{ab} \mathbf{z} - \mathbf{B}_a u)
\end{equation}

\subsection{Robustness to Noise}
The filtering strategy significantly improves robustness to measurement noise:
\begin{itemize}
    \item Reduces estimation error variance
    \item Prevents observer from tracking high-frequency sensor noise
    \item Improves closed-loop stability margins
    \item Reduces actuator chatter and energy consumption
\end{itemize}

\begin{figure}[h!]
    \centering
    \includegraphics[width=\columnwidth]{filtered_observer_block.png} % Adjusted for two-column layout
    \caption{Minimum-Order Observer with Low-Pass Filtering - Integrates measurement filtering with reduced-order estimation}
    \label{fig:filtered_observer}
</figure>

\section{Integrated Control-Observer System}

\subsection{Separation Principle}
\begin{equation}
    u = -\mathbf{K}\hat{\mathbf{x}}
\end{equation}

\subsection{Combined Closed-Loop Dynamics}
\begin{equation}
    \begin{bmatrix} \dot{\mathbf{x}} \\ \dot{\mathbf{e}} \end{bmatrix} =
    \begin{bmatrix} \mathbf{A} - \mathbf{B}\mathbf{K} & \mathbf{B}\mathbf{K} \\
                    0 & \mathbf{A} - \mathbf{L}\mathbf{C} \end{bmatrix}
    \begin{bmatrix} \mathbf{x} \\ \mathbf{e} \end{bmatrix}
\end{equation}

\subsection{Pole Placement Strategy}
For the ballbot system:
\begin{itemize}
    \item \textbf{Controller poles}: Placed by LQR at locations balancing stability and control effort
    \item \textbf{Observer poles}: Placed faster than controller poles (typically 2–3× faster) to ensure fast convergence of estimation errors before they significantly affect control
    \item \textbf{Time-scale separation}: Ensures that observer errors decay quickly relative to closed-loop response
\end{itemize}

\section{Summary of Observer Architectures}

\begin{table}[h!]
\centering
\begin{tabular}{|l|c|c|c|}
\hline
\textbf{Observer Type} & \textbf{Order} & \textbf{States} & \textbf{Advantages} \\ \hline
Full-Order Kalman & 4 & All & Complete state information, standard \\ \hline
Reduced-Order & 3 & \(\dot{x}, x, \dot{\theta}\) & Computational efficiency, no redundancy \\ \hline
Reduced with Filter & 3 & Filtered + unmeasured & Noise rejection, smooth estimates \\ \hline
\end{tabular}
\caption{Comparison of Observer Architectures}
\label{tab:observers}
\end{table}

\chapter{Root Locus Control}
Root Locus analysis is a graphical method used to examine how the closed-loop poles of a system vary with a change in a system parameter, typically a gain. This method helps in understanding the stability and performance of the system.

\section{Root Locus Controller Design (from Rlocus\_ctrl.m)}
The `Rlocus_ctrl.m` MATLAB script is used to design a controller based on Root Locus analysis. This script typically involves:
\begin{enumerate}
    \item Defining the open-loop transfer function of the system.
    \item Plotting the Root Locus using the `rlocus` function.
    \item Selecting an appropriate gain $K$ to place the closed-loop poles in desired locations, ensuring stability and desired transient response.
    \item Designing a compensator (e.g., lead, lag, PID) if simple proportional control is insufficient.
</enumerate}

\section{Simulink Model (Rlocus.slx)}
The Root Locus designed controller is simulated using the `Rlocus.slx` Simulink model. This model typically contains:
\begin{itemize}
    \item The transfer function representation of the plant.
    \item The controller block (e.g., a simple gain or a designed compensator).
    \item Feedback loop.
    \item Scopes to analyze the system's performance.
\end{itemize}

\begin{figure}[h!]
    \centering
    \includegraphics[width=\columnwidth]{placeholder_rlocus_plot.png} % Adjusted for two-column layout
    \caption{Placeholder for Root Locus Plot}
    \label{fig:rlocus_plot}
\end{figure}

\begin{figure}[h!]
    \centering
    \includegraphics[width=\columnwidth]{placeholder_rlocus_simulink.png} % Adjusted for two-column layout
    \caption{Placeholder for Root Locus Simulink Model Block Diagram}
    \label{fig:rlocus_simulink}
\end{figure}

\end{document}
